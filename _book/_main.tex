% Options for packages loaded elsewhere
\PassOptionsToPackage{unicode}{hyperref}
\PassOptionsToPackage{hyphens}{url}
%
\documentclass[
]{book}
\usepackage{amsmath,amssymb}
\usepackage{lmodern}
\usepackage{iftex}
\ifPDFTeX
  \usepackage[T1]{fontenc}
  \usepackage[utf8]{inputenc}
  \usepackage{textcomp} % provide euro and other symbols
\else % if luatex or xetex
  \usepackage{unicode-math}
  \defaultfontfeatures{Scale=MatchLowercase}
  \defaultfontfeatures[\rmfamily]{Ligatures=TeX,Scale=1}
\fi
% Use upquote if available, for straight quotes in verbatim environments
\IfFileExists{upquote.sty}{\usepackage{upquote}}{}
\IfFileExists{microtype.sty}{% use microtype if available
  \usepackage[]{microtype}
  \UseMicrotypeSet[protrusion]{basicmath} % disable protrusion for tt fonts
}{}
\makeatletter
\@ifundefined{KOMAClassName}{% if non-KOMA class
  \IfFileExists{parskip.sty}{%
    \usepackage{parskip}
  }{% else
    \setlength{\parindent}{0pt}
    \setlength{\parskip}{6pt plus 2pt minus 1pt}}
}{% if KOMA class
  \KOMAoptions{parskip=half}}
\makeatother
\usepackage{xcolor}
\usepackage{color}
\usepackage{fancyvrb}
\newcommand{\VerbBar}{|}
\newcommand{\VERB}{\Verb[commandchars=\\\{\}]}
\DefineVerbatimEnvironment{Highlighting}{Verbatim}{commandchars=\\\{\}}
% Add ',fontsize=\small' for more characters per line
\usepackage{framed}
\definecolor{shadecolor}{RGB}{248,248,248}
\newenvironment{Shaded}{\begin{snugshade}}{\end{snugshade}}
\newcommand{\AlertTok}[1]{\textcolor[rgb]{0.94,0.16,0.16}{#1}}
\newcommand{\AnnotationTok}[1]{\textcolor[rgb]{0.56,0.35,0.01}{\textbf{\textit{#1}}}}
\newcommand{\AttributeTok}[1]{\textcolor[rgb]{0.77,0.63,0.00}{#1}}
\newcommand{\BaseNTok}[1]{\textcolor[rgb]{0.00,0.00,0.81}{#1}}
\newcommand{\BuiltInTok}[1]{#1}
\newcommand{\CharTok}[1]{\textcolor[rgb]{0.31,0.60,0.02}{#1}}
\newcommand{\CommentTok}[1]{\textcolor[rgb]{0.56,0.35,0.01}{\textit{#1}}}
\newcommand{\CommentVarTok}[1]{\textcolor[rgb]{0.56,0.35,0.01}{\textbf{\textit{#1}}}}
\newcommand{\ConstantTok}[1]{\textcolor[rgb]{0.00,0.00,0.00}{#1}}
\newcommand{\ControlFlowTok}[1]{\textcolor[rgb]{0.13,0.29,0.53}{\textbf{#1}}}
\newcommand{\DataTypeTok}[1]{\textcolor[rgb]{0.13,0.29,0.53}{#1}}
\newcommand{\DecValTok}[1]{\textcolor[rgb]{0.00,0.00,0.81}{#1}}
\newcommand{\DocumentationTok}[1]{\textcolor[rgb]{0.56,0.35,0.01}{\textbf{\textit{#1}}}}
\newcommand{\ErrorTok}[1]{\textcolor[rgb]{0.64,0.00,0.00}{\textbf{#1}}}
\newcommand{\ExtensionTok}[1]{#1}
\newcommand{\FloatTok}[1]{\textcolor[rgb]{0.00,0.00,0.81}{#1}}
\newcommand{\FunctionTok}[1]{\textcolor[rgb]{0.00,0.00,0.00}{#1}}
\newcommand{\ImportTok}[1]{#1}
\newcommand{\InformationTok}[1]{\textcolor[rgb]{0.56,0.35,0.01}{\textbf{\textit{#1}}}}
\newcommand{\KeywordTok}[1]{\textcolor[rgb]{0.13,0.29,0.53}{\textbf{#1}}}
\newcommand{\NormalTok}[1]{#1}
\newcommand{\OperatorTok}[1]{\textcolor[rgb]{0.81,0.36,0.00}{\textbf{#1}}}
\newcommand{\OtherTok}[1]{\textcolor[rgb]{0.56,0.35,0.01}{#1}}
\newcommand{\PreprocessorTok}[1]{\textcolor[rgb]{0.56,0.35,0.01}{\textit{#1}}}
\newcommand{\RegionMarkerTok}[1]{#1}
\newcommand{\SpecialCharTok}[1]{\textcolor[rgb]{0.00,0.00,0.00}{#1}}
\newcommand{\SpecialStringTok}[1]{\textcolor[rgb]{0.31,0.60,0.02}{#1}}
\newcommand{\StringTok}[1]{\textcolor[rgb]{0.31,0.60,0.02}{#1}}
\newcommand{\VariableTok}[1]{\textcolor[rgb]{0.00,0.00,0.00}{#1}}
\newcommand{\VerbatimStringTok}[1]{\textcolor[rgb]{0.31,0.60,0.02}{#1}}
\newcommand{\WarningTok}[1]{\textcolor[rgb]{0.56,0.35,0.01}{\textbf{\textit{#1}}}}
\usepackage{longtable,booktabs,array}
\usepackage{calc} % for calculating minipage widths
% Correct order of tables after \paragraph or \subparagraph
\usepackage{etoolbox}
\makeatletter
\patchcmd\longtable{\par}{\if@noskipsec\mbox{}\fi\par}{}{}
\makeatother
% Allow footnotes in longtable head/foot
\IfFileExists{footnotehyper.sty}{\usepackage{footnotehyper}}{\usepackage{footnote}}
\makesavenoteenv{longtable}
\usepackage{graphicx}
\makeatletter
\def\maxwidth{\ifdim\Gin@nat@width>\linewidth\linewidth\else\Gin@nat@width\fi}
\def\maxheight{\ifdim\Gin@nat@height>\textheight\textheight\else\Gin@nat@height\fi}
\makeatother
% Scale images if necessary, so that they will not overflow the page
% margins by default, and it is still possible to overwrite the defaults
% using explicit options in \includegraphics[width, height, ...]{}
\setkeys{Gin}{width=\maxwidth,height=\maxheight,keepaspectratio}
% Set default figure placement to htbp
\makeatletter
\def\fps@figure{htbp}
\makeatother
\setlength{\emergencystretch}{3em} % prevent overfull lines
\providecommand{\tightlist}{%
  \setlength{\itemsep}{0pt}\setlength{\parskip}{0pt}}
\setcounter{secnumdepth}{5}
\usepackage{booktabs}
\ifLuaTeX
  \usepackage{selnolig}  % disable illegal ligatures
\fi
\usepackage[]{natbib}
\bibliographystyle{plainnat}
\IfFileExists{bookmark.sty}{\usepackage{bookmark}}{\usepackage{hyperref}}
\IfFileExists{xurl.sty}{\usepackage{xurl}}{} % add URL line breaks if available
\urlstyle{same} % disable monospaced font for URLs
\hypersetup{
  pdftitle={A Minimal Book Example},
  pdfauthor={John Doe},
  hidelinks,
  pdfcreator={LaTeX via pandoc}}

\title{A Minimal Book Example}
\author{John Doe}
\date{2022-08-09}

\begin{document}
\maketitle

{
\setcounter{tocdepth}{1}
\tableofcontents
}
\hypertarget{about}{%
\chapter{About}\label{about}}

This is a \emph{sample} book written in \textbf{Markdown}. You can use anything that Pandoc's Markdown supports; for example, a math equation \(a^2 + b^2 = c^2\).

\hypertarget{usage}{%
\section{Usage}\label{usage}}

Each \textbf{bookdown} chapter is an .Rmd file, and each .Rmd file can contain one (and only one) chapter. A chapter \emph{must} start with a first-level heading: \texttt{\#\ A\ good\ chapter}, and can contain one (and only one) first-level heading.

Use second-level and higher headings within chapters like: \texttt{\#\#\ A\ short\ section} or \texttt{\#\#\#\ An\ even\ shorter\ section}.

The \texttt{index.Rmd} file is required, and is also your first book chapter. It will be the homepage when you render the book.

\hypertarget{render-book}{%
\section{Render book}\label{render-book}}

You can render the HTML version of this example book without changing anything:

\begin{enumerate}
\def\labelenumi{\arabic{enumi}.}
\item
  Find the \textbf{Build} pane in the RStudio IDE, and
\item
  Click on \textbf{Build Book}, then select your output format, or select ``All formats'' if you'd like to use multiple formats from the same book source files.
\end{enumerate}

Or build the book from the R console:

\begin{Shaded}
\begin{Highlighting}[]
\NormalTok{bookdown}\SpecialCharTok{::}\FunctionTok{render\_book}\NormalTok{()}
\end{Highlighting}
\end{Shaded}

To render this example to PDF as a \texttt{bookdown::pdf\_book}, you'll need to install XeLaTeX. You are recommended to install TinyTeX (which includes XeLaTeX): \url{https://yihui.org/tinytex/}.

\hypertarget{preview-book}{%
\section{Preview book}\label{preview-book}}

As you work, you may start a local server to live preview this HTML book. This preview will update as you edit the book when you save individual .Rmd files. You can start the server in a work session by using the RStudio add-in ``Preview book'', or from the R console:

\begin{Shaded}
\begin{Highlighting}[]
\NormalTok{bookdown}\SpecialCharTok{::}\FunctionTok{serve\_book}\NormalTok{()}
\end{Highlighting}
\end{Shaded}

\hypertarget{manifolds}{%
\chapter{Manifolds}\label{manifolds}}

\hypertarget{introduction}{%
\section{Introduction}\label{introduction}}

Let \(M\) be a extcolor\{magenta\}\{second-countable\}\footnote{Arguably, the truly important property here is extcolor\{magenta\}\{paracompactness\}, which is slightly stronger and enables partitions of unity (enabling local-to-global promotions). However, it is a result that Hausdorff, second countable, extcolor\{magenta\}\{locally compact\} space is paracompact (and we get local compactness follows from locally Euclidean). Second countability also contributes to the feasibility of Euclidean embeddings and other nice, preferable behavior.

  References: \href{https://math.stackexchange.com/questions/2131530/why-is-important-for-a-manifold-to-have-countable-basis}{Second countability and manifolds}}, extcolor\{magenta\}\{Hausdorff\}\footnote{Hausdorff topological spaces feature points which are sufficiently disjoint: in particular, calculus depends upon limits, and Hausdorff \(\implies\) unique limits as desired (note, though, that the converse isn't true).}, extcolor\{magenta\}\{locally Euclidean topological space\} of dimension \(n\). We define an extcolor\{magenta\}\{equivalence relation\} on the set of homeomorphisms between extcolor\{magenta\}\{open\} subsets of \(M\) and \(\mathbb{R}^n\) given by \(\phi \sim \psi\) when \(\psi \circ \phi^{-1}\) is extcolor\{magenta\}\{smooth\}. We then choose a \(\mathcal{U} = \{(U_\alpha, \phi_\alpha)\}\) (i.e., \(\phi_\alpha : U_\alpha \to \mathbb{R}^n\)) such that the \(\{U_\alpha\}\) cover \(M\) and the \(\{\phi_\alpha\}\) are an equivalence class: this is denoted a extcolor\{blue\}\{maximal atlas\}\footnote{Definitions vary here (indeed, it is more conventional to merely require ``maximal'' atlases) but the general motivation is as follows: given a chart \(\phi\) on a manifold \(M\), there are likely uncountably many collections of charts covering \(M\) containing \(\phi\), but there is a \emph{unique} (i.e., canonical) choice of equivalence class of charts containing \(\phi\).

  References: \href{https://math.stackexchange.com/questions/66554/is-zorns-lemma-required-to-prove-the-existence-of-a-maximal-atlas-on-a-manifold}{Axiom of choice and maximal atlases}}. We then say that \(M\) is an \(n\)-dimensional extcolor\{blue\}\{smooth manifold\}\footnote{Our consideration of differential topology/geometry is motivated by physics, which interests itself in the dynamics (or change) of our universe. extcolor\{magenta\}\{Calculus\}, in a word, is the mathematics of change: hence, we are interested in studying the \emph{least structured} space that permits the calculus. This is not Euclidean space itself but rather a smooth manifold, a space that need only resemble Euclidean space \emph{locally}.} (or manifold). Let \((\phi, U) \in \mathcal{U}\): \(\phi\) is a extcolor\{blue\}\{coordinate chart\} (or chart) and the components of \(\phi\), \(x^i\) (i.e., \(\phi_\alpha(m) = (x^1(m), \dots, x^n(m))\)), are extcolor\{blue\}\{coordinates\}. We say real-valued maps are extcolor\{blue\}\{functions\} (e.g., the \(x^i\) are functions).

\hypertarget{smooth-maps}{%
\section{Smooth Maps}\label{smooth-maps}}

Given another manifold \(N\), we say \(f : V \to N\) is a extcolor\{blue\}\{smooth map\} (or smooth) for an open set \(V \subseteq M\) when for all \(m \in U\), there exist charts \(\phi\) and \(\psi\) defined around \(m\) and \(f(m)\) such that \(\psi \circ f \circ \phi^{-1}\) is smooth. For arbitrary \(U\), we say the same when there exists \(F : W \to N\) for an open set \(V \subset W \subseteq M\) such that \(F_{|V} = f\) and \(F\) is smooth. We call smooth maps with smooth inverse extcolor\{blue\}\{diffeomorphisms\}. We use \(C^\infty(M)\), \(\text{Diff}(M,N)\), and \(\text{Diff}(M)\) to denote the spaces of smooth functions on \(M\), diffeomorphisms \(M \to N\), and diffeomorphisms \(M \to M\), respectively. From this point forward, all maps are smooth unless otherwise specified.

\hypertarget{tangent-spaces}{%
\section{Tangent Spaces}\label{tangent-spaces}}

Let \(T_m M\) denote the extcolor\{magenta\}\{vector space\} of extcolor\{magenta\}\{linear derivations\} on the (vector) space of extcolor\{magenta\}\{germs\} of functions defined around \(m\), \(F_m\). Alternatively, let \(T_m M\) be the extcolor\{magenta\}\{quotient ring\} \((F_m/F_m^2)^*\), where \(*\) denotes the extcolor\{magenta\}\{dual space\}. \(T_m M\) has dimension \(n\), and we call it the extcolor\{blue\}\{tangent space\} to \(M\) at \(m\) and elements of \(T_m M\) extcolor\{blue\}\{vectors\}. There is a natural map \(f \mapsto f_*\) from the set of smooth functions \(M \to N\), denoted \(C^\infty(M,N)\), to the set of extcolor\{magenta\}\{endomorphisms\} \(T_m M \to T_{f(m)} N\) given by \(f_* X(g) \mapsto X(g \circ f)\) (where \(X \in T_m M\) and \(g \in C^\infty(M)\), the extcolor\{magenta\}\{ring\} of smooth functions on \(M\)). We call \(f_*\) the extcolor\{blue\}\{pushfoward\} of \(f\). We define \(T_m^* M\) to be the extcolor\{blue\}\{cotangent space\} to \(M\) at \(m\); there is a natural map \(d : C^\infty(M) \to T_m^* M\) given by \(f \mapsto df(m) = v \mapsto v(f)\), which we call the extcolor\{blue\}\{differential\}. We also have the dual map \(f \mapsto f^*\), the extcolor\{blue\}\{pullback\}, acting as \(T_{f(m)}^* N \to T_m^* M\) by \(f^* A(X) = A(f_* X)\). Given a chart \(\phi\) around \(M\), a basis for \(T_m M\) is given by \(\frac{\partial}{\partial x^i}\) or \(\partial_{i}\), given by
\begin{equation}     
    \partial_{i}f = \frac{\partial (f \circ \phi^{-1})}{\partial r^i}\Big|_m 
\end{equation}
where \(r^i\) is the \(i\)th Euclidean coordinate. A basis is also given for \(T^*_m M\) by the \(dx^i\). Finally, we define the extcolor\{blue\}\{tangent bundle\} \(TM = \cup_{m\in M} T_m M\) and the extcolor\{blue\}\{cotangent bundle\} \(T^*M = \cup_{m\in M} T_m^* M\); both are \(2n\)-dimensional smooth manifolds equipped with natural projection maps onto \(M\).

\hypertarget{fibre-bundles}{%
\chapter{Fibre Bundles}\label{fibre-bundles}}

\hypertarget{lie-theory}{%
\chapter{Lie Theory}\label{lie-theory}}

Example (short) footnote\footnote{blah blah blah}.

Example (long) footnote\footnote{blaher blaher blaher}

\hypertarget{applications}{%
\chapter{Applications}\label{applications}}

Some \emph{significant} applications are demonstrated in this chapter.

\hypertarget{example-one}{%
\section{Example one}\label{example-one}}

\hypertarget{example-two}{%
\section{Example two}\label{example-two}}

\hypertarget{complex-manifolds}{%
\chapter{Complex Manifolds}\label{complex-manifolds}}

We have finished a nice book.

  \bibliography{book.bib,packages.bib}

\end{document}
