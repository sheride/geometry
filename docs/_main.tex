% Options for packages loaded elsewhere
\PassOptionsToPackage{unicode}{hyperref}
\PassOptionsToPackage{hyphens}{url}
%
\documentclass[
]{book}
\usepackage{amsmath,amssymb}
\usepackage{lmodern}
\usepackage{iftex}
\ifPDFTeX
  \usepackage[T1]{fontenc}
  \usepackage[utf8]{inputenc}
  \usepackage{textcomp} % provide euro and other symbols
\else % if luatex or xetex
  \usepackage{unicode-math}
  \defaultfontfeatures{Scale=MatchLowercase}
  \defaultfontfeatures[\rmfamily]{Ligatures=TeX,Scale=1}
\fi
% Use upquote if available, for straight quotes in verbatim environments
\IfFileExists{upquote.sty}{\usepackage{upquote}}{}
\IfFileExists{microtype.sty}{% use microtype if available
  \usepackage[]{microtype}
  \UseMicrotypeSet[protrusion]{basicmath} % disable protrusion for tt fonts
}{}
\makeatletter
\@ifundefined{KOMAClassName}{% if non-KOMA class
  \IfFileExists{parskip.sty}{%
    \usepackage{parskip}
  }{% else
    \setlength{\parindent}{0pt}
    \setlength{\parskip}{6pt plus 2pt minus 1pt}}
}{% if KOMA class
  \KOMAoptions{parskip=half}}
\makeatother
\usepackage{xcolor}
\usepackage{longtable,booktabs,array}
\usepackage{calc} % for calculating minipage widths
% Correct order of tables after \paragraph or \subparagraph
\usepackage{etoolbox}
\makeatletter
\patchcmd\longtable{\par}{\if@noskipsec\mbox{}\fi\par}{}{}
\makeatother
% Allow footnotes in longtable head/foot
\IfFileExists{footnotehyper.sty}{\usepackage{footnotehyper}}{\usepackage{footnote}}
\makesavenoteenv{longtable}
\usepackage{graphicx}
\makeatletter
\def\maxwidth{\ifdim\Gin@nat@width>\linewidth\linewidth\else\Gin@nat@width\fi}
\def\maxheight{\ifdim\Gin@nat@height>\textheight\textheight\else\Gin@nat@height\fi}
\makeatother
% Scale images if necessary, so that they will not overflow the page
% margins by default, and it is still possible to overwrite the defaults
% using explicit options in \includegraphics[width, height, ...]{}
\setkeys{Gin}{width=\maxwidth,height=\maxheight,keepaspectratio}
% Set default figure placement to htbp
\makeatletter
\def\fps@figure{htbp}
\makeatother
\setlength{\emergencystretch}{3em} % prevent overfull lines
\providecommand{\tightlist}{%
  \setlength{\itemsep}{0pt}\setlength{\parskip}{0pt}}
\setcounter{secnumdepth}{5}
\usepackage{booktabs}
\ifLuaTeX
  \usepackage{selnolig}  % disable illegal ligatures
\fi
\usepackage[]{natbib}
\bibliographystyle{plainnat}
\IfFileExists{bookmark.sty}{\usepackage{bookmark}}{\usepackage{hyperref}}
\IfFileExists{xurl.sty}{\usepackage{xurl}}{} % add URL line breaks if available
\urlstyle{same} % disable monospaced font for URLs
\hypersetup{
  pdftitle={Geometry},
  pdfauthor={Elijah Sheridan},
  hidelinks,
  pdfcreator={LaTeX via pandoc}}

\title{Geometry}
\author{Elijah Sheridan}
\date{2022-08-10}

\begin{document}
\maketitle

{
\setcounter{tocdepth}{1}
\tableofcontents
}
\hypertarget{intro}{%
\chapter{Introduction}\label{intro}}

What follows endeavors to encapsulate the author's knowledge of geometry (and the notions upon which it depends, which decidedly extend well beyond the boundaries of geometry itself) as he studies theoretical particle physics and string theory.

\hypertarget{manifolds}{%
\chapter{Manifolds}\label{manifolds}}

\hypertarget{construction}{%
\section{Construction}\label{construction}}

Let \(M\) be a extcolor\{magenta\}\{second-countable\}\footnote{Arguably, the truly important property here is extcolor\{magenta\}\{paracompactness\}, which is slightly stronger and enables partitions of unity (enabling local-to-global promotions).
  However, it is a result that Hausdorff, second countable, extcolor\{magenta\}\{locally compact\} space is paracompact (and we get local compactness follows from locally Euclidean).
  Second countability also contributes to the feasibility of Euclidean embeddings and other nice, preferable behavior.

  References: \href{https://math.stackexchange.com/questions/2131530/why-is-important-for-a-manifold-to-have-countable-basis}{Second countability and manifolds}}, extcolor\{magenta\}\{Hausdorff\}\footnote{Hausdorff topological spaces feature points which are sufficiently disjoint: in particular, calculus depends upon limits, and Hausdorff \(\implies\) unique limits as desired (note, though, that the converse isn't true).}, extcolor\{magenta\}\{locally Euclidean topological space\} of dimension \(n\).
We define an extcolor\{magenta\}\{equivalence relation\} on the set of homeomorphisms between extcolor\{magenta\}\{open\} subsets of \(M\) and \(\mathbb{R}^n\) given by \(\phi \sim \psi\) when \(\psi \circ \phi^{-1}\) is extcolor\{magenta\}\{smooth\}.
We then choose a \(\mathcal{U} = \{(U_\alpha, \phi_\alpha)\}\) (i.e., \(\phi_\alpha : U_\alpha \to \mathbb{R}^n\)) such that the \(\{U_\alpha\}\) cover \(M\) and the \(\{\phi_\alpha\}\) are an equivalence class: this is denoted a extcolor\{blue\}\{maximal atlas\}\footnote{Definitions vary here (indeed, it is more conventional to merely require ``maximal'' atlases) but the general motivation is as follows: given a chart \(\phi\) on a manifold \(M\), there are likely uncountably many collections of charts covering \(M\) containing \(\phi\), but there is a \emph{unique} (i.e., canonical) choice of equivalence class of charts containing \(\phi\).

  References: \href{https://math.stackexchange.com/questions/66554/is-zorns-lemma-required-to-prove-the-existence-of-a-maximal-atlas-on-a-manifold}{Axiom of choice and maximal atlases}}.
We then say that \(M\) is an \(n\)-dimensional extcolor\{blue\}\{smooth manifold\}\footnote{Our consideration of differential topology/geometry is motivated by physics, which interests itself in the dynamics (or change) of our universe. extcolor\{magenta\}\{Calculus\}, in a word, is the mathematics of change: hence, we are interested in studying the \emph{least structured} space that permits the calculus.
  This is not Euclidean space itself but rather a smooth manifold, a space that need only resemble Euclidean space \emph{locally}.} (or manifold).
Let \((\phi, U) \in \mathcal{U}\): \(\phi\) is a extcolor\{blue\}\{coordinate chart\} (or chart) and the components of \(\phi\), \(x^i\) (i.e., \(\phi_\alpha(m) = (x^1(m), \dots, x^n(m))\)), are extcolor\{blue\}\{coordinates\}.
We say real-valued maps are extcolor\{blue\}\{functions\} (e.g., the \(x^i\) are functions).

\hypertarget{smooth-maps}{%
\section{Smooth Maps}\label{smooth-maps}}

Given another manifold \(N\), we say \(f : V \to N\) is a extcolor\{blue\}\{smooth map\} (or smooth) for an open set \(V \subseteq M\) when for all \(m \in U\), there exist charts \(\phi\) and \(\psi\) defined around \(m\) and \(f(m)\) such that \(\psi \circ f \circ \phi^{-1}\) is smooth.
Given \(f : U \to N\) for arbitrary \(U \subset M\), we say the same when \(f\) is the restriction of a smooth map on some open \(W \supseteq V\).
We call smooth maps with smooth inverse extcolor\{blue\}\{diffeomorphisms\}.
We use \(C^\infty(M)\), \(\text{Diff}(M,N)\), and \(\text{Diff}(M)\) to denote the spaces of smooth functions on \(M\), diffeomorphisms \(M \to N\), and diffeomorphisms \(M \to M\), respectively.
From this point forward, all maps are smooth unless otherwise specified.

\hypertarget{tangent-spaces}{%
\section{Tangent Spaces}\label{tangent-spaces}}

Let \(T_m M\) denote the extcolor\{magenta\}\{vector space\} of extcolor\{magenta\}\{linear derivations\} on the (vector) space of extcolor\{magenta\}\{germs\} of functions defined around \(m\), \(F_m\).
Equivalently, let \(T_m M\) be the extcolor\{magenta\}\{quotient ring\} \((F_m/F_m^2)^*\), where \(*\) denotes the extcolor\{magenta\}\{dual space\}\footnote{TODO: prove equivalence of definitions.}.
\(T_m M\) has dimension \(n\), and we call it the extcolor\{blue\}\{tangent space\} to \(M\) at \(m\) and elements of \(T_m M\) extcolor\{blue\}\{vectors\}.
There is a natural map \(f \mapsto f_*\) from the set of smooth functions \(M \to N\), denoted \(C^\infty(M,N)\), to the set of extcolor\{magenta\}\{endomorphisms\} \(T_m M \to T_{f(m)} N\) given by \(f_* X(g) \mapsto X(g \circ f)\) (where \(X \in T_m M\) and \(g \in C^\infty(M)\), the extcolor\{magenta\}\{ring\} of smooth functions on \(M\)). W
e call \(f_*\) the extcolor\{blue\}\{pushfoward\} of \(f\).
We define \(T_m^* M\) to be the extcolor\{blue\}\{cotangent space\} to \(M\) at \(m\), and we have the dual map \(f \mapsto f^*\), the extcolor\{blue\}\{pullback\}, acting as \(T_{f(m)}^* N \to T_m^* M\) by \(f^* A(X) = A(f_* X)\).
There is additionally a natural map \(d : C^\infty(M) \to T_m^* M\) given by \(f \mapsto df(m) = v \mapsto v(f)\), which we call the extcolor\{blue\}\{differential\}.
Given a chart \(\phi\) around \(M\), a basis for \(T_m M\) is given by \(\frac{\partial}{\partial x^i}\) or \(\partial_{i}\), given by
\begin{equation}     
    \partial_{i}f = \frac{\partial (f \circ \phi^{-1})}{\partial r^i}\Big|_m 
\end{equation}
where \(r^i\) is the \(i\)th Euclidean coordinate.
A basis is also given for \(T^*_m M\) by the \(dx^i\).
Finally, we define the extcolor\{blue\}\{tangent bundle\} \(TM = \cup_{m\in M} T_m M\) and the extcolor\{blue\}\{cotangent bundle\} \(T^*M = \cup_{m\in M} T_m^* M\); both are \(2n\)-dimensional smooth manifolds equipped with natural projection maps onto \(M\).

\hypertarget{fibre-bundles}{%
\chapter{Fibre Bundles}\label{fibre-bundles}}

\hypertarget{construction-1}{%
\section{Construction}\label{construction-1}}

Let \(M, F\) be manifolds and \(E\) be a extcolor\{blue\}\{fibre bundle\} with extcolor\{blue\}\{base\} \(M\) and extcolor\{blue\}\{fibre\} \(F\), or a manifold endowed with a surjective projection \(\pi : E \to M\) such that \(M\) admits an open covering \(\{U_\alpha\}\) for which each \(U_\alpha\) has a diffeomorphism \(\phi_\alpha : \pi^{-1}(U_\alpha) \to U_\alpha \times F\) acting by \(p \to (\pi(p), \xi_\alpha(p))\) for \(p \in P\) and some \(\xi_\alpha : U_\alpha \to \text{Diff}(F)\). This implies \(\pi^{-1}(m)\) is diffeomorphic to \(F\); we say \(E\) is locally a product of \(M\) and \(F\) and that the \((U_\alpha, \phi_\alpha)\) is a extcolor\{blue\}\{local trivialization\}. On \(U_\alpha \cap U_\beta\) we have functions \(\phi_{\alpha\beta} = \phi_\alpha \circ \phi_\beta^{-1} : (U_\alpha \cap U_\beta) \times F \to (U_\alpha \cap U_\beta) \times F\) given by \((m, x) \mapsto (m, \xi_{\alpha\beta}(m)(x))\) for some \(\xi_{\alpha\beta}(m) \in \text{Diff}(F)\) called extcolor\{blue\}\{transition functions\}. Sometimes we require \(\xi_{\alpha\beta}(m) \in G\), a extcolor\{blue\}\{topological group\} acting on \(F\) on the left by diffeomorphisms (i.e., a subgroup of \(\text{Diff}(F)\)). If a fibre bundle's local trivializations satisfy this maximally, we say \(E\) is a extcolor\{blue\}\{\(G\)-bundle\}, and that \(G\) is the extcolor\{blue\}\{structure group\}. We note that \(\xi_{\alpha\alpha} = 1\), \(\xi_{\alpha\beta} = \xi_{\beta\alpha}^{-1}\), and the extcolor\{blue\}\{cocycle condition\} \(\xi_{\alpha\beta} \circ \xi_{\beta\delta} \circ \xi_{\delta\alpha} = 1\) holds on triple overlaps. Fibre bundles are uniquely determined by the base manifold and the transition functions. Given a manifold \(N\) and a map \(g : N \to M\), the pullback bundle \(g^*E\) is the subset of \(N \times E\) such that \(g \circ \text{proj}_{1} = \pi \circ \text{proj}_{2}\) with projection \(\text{proj}_{1} : g^*E \to N\). A extcolor\{blue\}\{vector bundle\} is a fibre bundle whose fibres are vector spaces and whose local trivializations are fibre-wise linear isomorphisms.

\hypertarget{lie-groups}{%
\section{Lie Groups}\label{lie-groups}}

A extcolor\{blue\}\{Lie group\} \(G\) is a smooth manifold with group structure such that the binary operation is smooth.\footnote{Some authors require the inverse map \(g \mapsto g^{-1}\) to be smooth as well, but this follows from the smoothness of the binary operation.}
Elements \(g \in G\) induce \(R_g, L_g, A_g \in \text{Diff}(G)\) by \(R_g : h \mapsto hg\), \(L_g : h \mapsto gh\), and \(A_g = L_g \circ R_{g^{-1}} : h \mapsto ghg^{-1}\).

\hypertarget{principal-bundles}{%
\section{Principal Bundles}\label{principal-bundles}}

Let \(G\) be a Lie group and \(P\) be a extcolor\{blue\}\{principle \(G\)-bundle\} with base \(M\), or a \(G\)-bundle over \(M\) with fibre \(G\) and transition functions given by left multiplication. Because left and right multiplication commute, we have an invariant right action of \(G\) on \(P\)\footnote{If \(p \in \pi^{-1}(U_\alpha)\) and \(\phi_\alpha(p) = (m,h)\), then \(pg = \phi_\alpha^{-1}(m,hg)\).}. Equivalently, a principle \(G\)-bundle \(P\) is a fibre bundle with a extcolor\{magenta\}\{regular\} smooth right action by a Lie group \(G\) that preserves fibres and the \(\xi_\alpha\) are \(G\)-equivariant. It follows that \(P/G = M\)\footnote{The second definition admits an exchange between fibre-preservation and \(P/G = M\)} and \(E\) admits a local trivialization with \(M \times G\).

\hypertarget{associated-bundles}{%
\section{Associated Bundles}\label{associated-bundles}}

\hypertarget{sections}{%
\section{Sections}\label{sections}}

\hypertarget{vector-fields}{%
\subsection{Vector Fields}\label{vector-fields}}

\hypertarget{tensor-fields}{%
\subsection{Tensor Fields}\label{tensor-fields}}

\hypertarget{differential-forms}{%
\subsection{Differential Forms}\label{differential-forms}}

\hypertarget{lie-theory}{%
\chapter{Lie Theory}\label{lie-theory}}

\hypertarget{lie-algebras}{%
\section{Lie Algebras}\label{lie-algebras}}

Let \(G\) be a Lie group.
A vector field \(X\) on \(G\) satisfying \((L_g)_* X = X \circ L_g\) is called extcolor\{blue\}\{left invariant\}.
The extcolor\{blue\}\{Lie algebra\} to \(G\), \(\mathfrak{g}\), is the set of all left-invariant vector fields on \(G\); the name is natural as the Lie bracket induces a Lie algebraic structure on this set.
Note that \(\mathfrak{g}\) is naturally isomorphic to \(T_e G\), where \(e\) is the identity element, as \(Y \in T_e G\) induces a vector field \(X\) given by \(X_g = (L_g)_* Y\); in particular, this means \(\dim \mathfrak{g} = \dim G\).
Given a basis \(X_1, \dots, X_n\) of \(\mathfrak{g}\), we have that \([X_i, X_j] = c_{ij}^h X_h\) and we say the \(c_{ij}^h\) are the extcolor\{blue\}\{structure constants\} associated with the basis.
Identifying \(\mathfrak{g} \cong T_e G\) there is a natural \(\mathfrak{g}\)-valued one-form \(\theta\) on \(G\) defined by \(v \mapsto (L_{g^{-1}})_*(v)\) for \(v \in T_g G\).
We call this the extcolor\{blue\}\{Maurer-Cartan one-form\}.
Noting that, if \(G\) is a matrix Lie group, \((L_g)_*\) coincides with the natural matrix multiplication action of the matrix \(g\) on \(TG\), we have that \(\theta(v) = g^{-1}v\) where the right hand side is matrix multiplication\footnote{If we let \(G\) be embedded in a matrix Lie group by a map \(\phi\), then the Maurer-Carten form on \(\phi(G)\) is \(\theta = \phi(g^{-1})\phi_*\).}.

A extcolor\{blue\}\{one-parameter subgroup\} on \(G\) is a continuous group homomorphism \(\mathbb{R} \to G\). For \(X \in \mathfrak{g}\), let \(\phi_{X,t} : G \to G\) be the associated flow, and let \(g_X(t) = \phi_{X,t}(e) \in G\); then \(g_X\) is a one parameter subgroup on \(G\) (i.e., \(g_X(t)g_X(s) = g_S(t+s)\)).
Moreover, we have the extcolor\{blue\}\{exponential map\} \(\exp : \mathfrak{g} \to G\) given by \(X \mapsto g_X(1)\).
From this it follows that \(g_X(t) = \exp(tX)\); indeed, this is the most general form for a one parameter group.

TODO: adjoint representation and Maurer-Cartan structure formula

\hypertarget{connections}{%
\chapter{Connections}\label{connections}}

To each \(A \in \mathfrak{g}\) we can naturally associate a extcolor\{blue\}\{fundamental vector field\} \(A^*\) on \(P\) given by \((A^\#)(p) = (\sigma_p)_*(A)\), where \(\sigma_p : G \to P\) is the map \(g \mapsto pg\). Equivalently, \((A^\#)(p)\) is the tangent vector to the curve \((\sigma_p \circ \exp)(At)\) at \(t = 0\). Define \(V_p = T_p \, \pi^{-1}(p) = \ker(\pi_*) \cap T_p P\), the extcolor\{blue\}\{vertical subspace\} of \(T_p P\); \(A \mapsto (A^*)_p\) is an isomorphism \(\mathfrak{g} \mapsto V_p\). Moreover, \(A \mapsto A^\#\) is equivariant with respect to the adjoint and principal actions of \(G\) and preserves the respective Lie brackets. The extcolor\{blue\}\{vertical bundle\} \(VP = TP\) is the vector subbundle of these vertical subspaces.

To a principal bundle we can associate an extcolor\{blue\}\{Atiyah sequence\}

where \(\iota\) is the natural inclusion and \(\overline{\pi}\) is the map \(X \mapsto (\pi_{TP}(X), (\pi_P)_*(X))\). A extcolor\{blue\}\{connection\} on \(P\) is a choice of equivariant split for this sequence (direct sum, left, and right splits are equivalent). Explicitly, the \(G\) actions are the diagonal action (identifying \(VP \cong P \times \mathfrak{g}\)), the canonical action, and the induced action from \(P \times TM\) leaving \(TM\) invariant, respectively.

In particular, a direct sum split \(\varphi\) is frequently identified with the vector subbundle \(HP = \varphi^{-1}(0 \oplus \pi^* TM)\), which is complementary to \(VP\) and invariant under \((R_g)_*\) for \(g \in G\) and referred to as an extcolor\{blue\}\{Ehresmann connection\}. Left splits can be understood as \(G\)-equivariant \(\mathfrak{g}\)-valued one-forms on \(P\), commonly referred to as extcolor\{blue\}\{connection one-forms\}. Right splits are interpreted as maps sending a vector in \(T_m M\) to one in any \(T_p P\) for \(p \in \pi^{-1}(m)\) and are known as extcolor\{blue\}\{horizontal lifts\}.

\hypertarget{complex-manifolds}{%
\chapter{Complex Manifolds}\label{complex-manifolds}}

We have finished a nice book.

\hypertarget{spin-geometry}{%
\chapter{Spin Geometry}\label{spin-geometry}}

\hypertarget{references}{%
\chapter*{References}\label{references}}
\addcontentsline{toc}{chapter}{References}

\hypertarget{appendix-appendix}{%
\appendix}


\hypertarget{tensors}{%
\chapter{Tensors}\label{tensors}}

\hypertarget{algebraic-topology}{%
\chapter{Algebraic Topology}\label{algebraic-topology}}

  \bibliography{book.bib,packages.bib}

\end{document}
